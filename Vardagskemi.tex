\documentclass[12pt]{article}
\usepackage{mathptmx}%http://ctan.org/pkg/mathptmx
\usepackage[utf8]{inputenc}% UTF-8
\usepackage{fancyhdr}
\usepackage[version=3]{mhchem}
\linespread{1.25}
\title{Materia och kemisk bindning - Projekt MI Vardagskemi \\ Kemi 1 \\ Ingert Krook}
\author{Mattias Smedman}


% Sidnummrering i högre hörnet
\pagestyle{fancy}
\fancyhead{}
\fancyfoot{}
\fancyfoot[R]{\thepage}

\begin{document}

	\pagenumbering{gobble}
	\maketitle
	\newpage
	\section*{Sammanfattning}
	\newpage
	\tableofcontents
	\newpage
	\pagenumbering{arabic}
	
	\section{Inledning}
		\subsection{Syfte och problemformulering}
			\paragraph{}Syftet med rapporten är att ta reda på generell information om tandkräm. Vem uppfann tandkrämen? Vilka ämnen finns i tandkräm? Varför finns dessa ämnen i tandkrämen?
			
		\subsection{Material och källkritik}
			\paragraph{}Illustrerad Vetenskap (2008) används som källa för historiken bakom tandkrämen. Källan är trovärdig då det är en av de större tidningarna om vetenskap.

		\subsection{Metod}
			\paragraph{}En litteraturstudie kommer genomföras för att besvara frågeställningarna. litteraturstudie har valts som metod då det är svårt att testa sig fram på gymnasienivå varför de olika ämnena finns i tandkräm och det blir lättast att jämföra olika sorters tandkrämers innehåll.

	\section{Tandkrämens Historia}
				\paragraph{}Enligt Illustrerad Vetenskap (2008) har tandkräm använts sedan 5000 år f Kr då egyptierna använde det i pulverform. Pulveret olika slipmedel såsom krossade oxhuvuden, pimpsten, krossade äggskal samt myrra. Hur detta pulver använts är skribenten inte säker på men nämner att arkeologer tror att egyptierna rengjorde sin mun genom att gnugga pulveret mot tänderna med fingrarna.\\
			
			\paragraph{}Illustrerad Vetenskap berättar även att forntidens romare och greker använde krossat träkol och bark som tandkräm. Enligt skribenten började romerna också att bekämpa dålig andedräkt genom att tillsätta kryddrötter och att efter det så utvecklades inte tandkrämen på de följande 1200 åren. \\
			
			Illusterad Vetenskap skriver att det var först vid år 1700 som europeiska och amerikanska läkare, tandläkare och kemister började utveckla moderna former av tandpulver. Då var Natriumbikarbonat huvudingrediensen i de flesta blandningarna och många av dem var skadliga, till exempel pulver av tegel. \\
			
			Skribenten berättar att man i början av 1800-talet så fick man idén att tillsätta glycerin till tandpulvert vilket gjorde att det blev tandkräm istället för pulver, det fick även blandningen att bli mer välsmakande. De berättar även att år 1873 lanserades Colgate, den första massproducerade tandkrämen och att den såldes på burk. Från och med 1892 kunde man köpa den på tuben såsom vi gör i dag.
	
	\section{Tandkrämens innehåll}
		\paragraph{}Enligt Svante Twetman (2015) så är fluor det viktigaste ämnet i tandkräm och de vanligaste fluoröreningarna är natriumfluorid (\ce{HF + NaOH -> NaF + H2O}) och natriummonofluorfosfat (\ce{PO3F^2- + OH^ -> HPO4^2- + F^-}) eller någon kombination av dessa. Han skriver även att det förekommer tandkräm som innehåller aminfluorid(\ce{C27H60F2N2O3}) och tennfluorid(\ce{SnF2}) men att det är i begränsad omfattning i Sverige. \\
		
		Twetman skriver även att tandkräm innehåller andra rengörande subsanser som slipmedel, blekningsmedel och tvålämnen och att det tandkräm kan innehålla enzymer och antibakteriella ämnen som motverkar tandsten.
	
		\paragraph{}Enligt TomsOfMaine är kiseldioxid(\ce{SiO2}) en del av tandkräm. Enligt skribenten är sand och obsidian vanliga former av kiseldioxid och att när kiseldioxiden är hydratiserad så är det ett luktfritt och smaklöst vit gelatinöst ämne som är kemiskt inert. 

	
	\section{Varför finns fluor i tandkräm?}
		\paragraph{}Enligt Twetman så uppstår kariesangrepp då tandens hårdvävnader förlorar des mineralsalter (demineraliseras) och att om tandens omgivning har ett lågt pH-värde så demineraliseras tanden genom att kalcium och fosfat löses ut. Han skriver att vid ett neutralt pH och en övermättad omgivning så kan mineralerna återinlagras och att fluor verkar i huvudsak lokalt på tandytan genom att få upplösningen att minskas och mineral inlagringen att öka. 

	\section{Varför finns kiseldioxid i tandkräm?}
		\paragraph{}Enligt TomsOfMaine ger hydriteserad kiseldioxid en lenare gelatinös kvalitet och att när det kombineras med 

	\section{Slutsats}
		\subsection{Sammanfattande diskussion / Slutsats}
			\paragraph{}I slutsats är tandkrämens historia mycket gammal. Den har utvecklats i över 5000 år och har gått från att vara bara en sorts slipmedel till att vara rötter och till sedan att vara ett pulver som man sedan fick idén att lägga till glycerin till pulvert för att få det till att bli en kräm såsom vi känner det idag. 

	\newpage
	\section{Källförteckning}
	http://illvet.se/teknologi/vem-uppfann-tandkramen\\
	https://en.wikipedia.org/wiki/Toothpaste\\
	http://www.internetodontologi.se/dyn\_main.asp?page=104\\
	http://www.tomsofmaine.com/ingredients/overlay/hydrated-silica\\
	\newpage
	\section{Appendix}




\end{document}